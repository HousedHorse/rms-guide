\documentclass[]{article}
\usepackage{lmodern}
\usepackage{amssymb,amsmath}
\usepackage{ifxetex,ifluatex}
\usepackage{fixltx2e} % provides \textsubscript
\ifnum 0\ifxetex 1\fi\ifluatex 1\fi=0 % if pdftex
  \usepackage[T1]{fontenc}
  \usepackage[utf8]{inputenc}
\else % if luatex or xelatex
  \ifxetex
    \usepackage{mathspec}
  \else
    \usepackage{fontspec}
  \fi
  \defaultfontfeatures{Ligatures=TeX,Scale=MatchLowercase}
\fi
% use upquote if available, for straight quotes in verbatim environments
\IfFileExists{upquote.sty}{\usepackage{upquote}}{}
% use microtype if available
\IfFileExists{microtype.sty}{%
\usepackage{microtype}
\UseMicrotypeSet[protrusion]{basicmath} % disable protrusion for tt fonts
}{}
\usepackage[margin=1in]{geometry}
\usepackage{hyperref}
\hypersetup{unicode=true,
            pdftitle={Random Map Scripting for AoE II HD},
            pdfauthor={William ``HousedHorse'' Findlay},
            pdfborder={0 0 0},
            breaklinks=true}
\urlstyle{same}  % don't use monospace font for urls
\usepackage{longtable,booktabs}
\usepackage{graphicx,grffile}
\makeatletter
\def\maxwidth{\ifdim\Gin@nat@width>\linewidth\linewidth\else\Gin@nat@width\fi}
\def\maxheight{\ifdim\Gin@nat@height>\textheight\textheight\else\Gin@nat@height\fi}
\makeatother
% Scale images if necessary, so that they will not overflow the page
% margins by default, and it is still possible to overwrite the defaults
% using explicit options in \includegraphics[width, height, ...]{}
\setkeys{Gin}{width=\maxwidth,height=\maxheight,keepaspectratio}
\IfFileExists{parskip.sty}{%
\usepackage{parskip}
}{% else
\setlength{\parindent}{0pt}
\setlength{\parskip}{6pt plus 2pt minus 1pt}
}
\setlength{\emergencystretch}{3em}  % prevent overfull lines
\providecommand{\tightlist}{%
  \setlength{\itemsep}{0pt}\setlength{\parskip}{0pt}}
\setcounter{secnumdepth}{5}

%%% Use protect on footnotes to avoid problems with footnotes in titles
\let\rmarkdownfootnote\footnote%
\def\footnote{\protect\rmarkdownfootnote}

%%% Change title format to be more compact
\usepackage{titling}

% Create subtitle command for use in maketitle
\newcommand{\subtitle}[1]{
  \posttitle{
    \begin{center}\large#1\end{center}
    }
}

\setlength{\droptitle}{-2em}

  \title{Random Map Scripting for AoE II HD}
    \pretitle{\vspace{\droptitle}\centering\huge}
  \posttitle{\par}
    \author{William ``HousedHorse'' Findlay}
    \preauthor{\centering\large\emph}
  \postauthor{\par}
      \predate{\centering\large\emph}
  \postdate{\par}
    \date{\today}

\usepackage{float}
\usepackage{listings}
\usepackage[hang,bf]{caption}
\usepackage{framed}
\usepackage[section]{placeins}

\allowdisplaybreaks

% fancy headers/footers
\makeatletter
\usepackage{fancyhdr}
\lhead{\@author}
\chead{}
\rhead{\@title}
\lfoot{}
\cfoot{\thepage}
\rfoot{}
\renewcommand{\headrulewidth}{0.4pt}
\pagestyle{plain}
\pagestyle{fancy}

\usepackage{amsmath, amsfonts,amssymb, amsthm}
\usepackage{siunitx}

\usepackage{setspace}
\usepackage{changepage}
\usepackage{titlesec}
\usepackage{aliascnt}

\floatplacement{figure}{!htb}
\floatplacement{table}{!htb}
\floatplacement{listing}{!htb}
\lstset{mathescape=true,numbers=left,breaklines=true,frame=single,language=C,captionpos=b,abovecaptionskip={\abovecaptionskip},belowcaptionskip={\belowcaptionskip},aboveskip=\intextsep}
\setlength{\captionmargin}{1in}

\newgeometry{margin=1in}

\newtheoremstyle{plain}
{12pt}   % ABOVESPACE
{12pt}   % BELOWSPACE
{\itshape}  % BODYFONT
{0pt}       % INDENT (empty value is the same as 0pt)
{\bfseries} % HEADFONT
{.}         % HEADPUNCT
{5pt plus 1pt minus 1pt} % HEADSPACE
{}          % CUSTOM-HEAD-SPEC

\newtheoremstyle{definition}
{12pt}   % ABOVESPACE
{12pt}   % BELOWSPACE
{\normalfont}  % BODYFONT
{0pt}       % INDENT (empty value is the same as 0pt)
{\bfseries} % HEADFONT
{.}         % HEADPUNCT
{5pt plus 1pt minus 1pt} % HEADSPACE
{}          % CUSTOM-HEAD-SPEC

\newtheoremstyle{remark}
{12pt}   % ABOVESPACE
{12pt}   % BELOWSPACE
{\normalfont}  % BODYFONT
{0pt}       % INDENT (empty value is the same as 0pt)
{\itshape} % HEADFONT
{.}         % HEADPUNCT
{5pt plus 1pt minus 1pt} % HEADSPACE
{}          % CUSTOM-HEAD-SPEC

\theoremstyle{plain}

% define theorem
\newtheorem{theorem}{Theorem}[section]
\providecommand*{\theoremautorefname}{Theorem}

% define lemma
\newtheorem{lemma}{Lemma}[section]
\providecommand*{\lemmaautorefname}{Lemma}

% define claim
\newtheorem{claim}{Claim}[section]
\providecommand*{\claimautorefname}{Claim}

% define corollary
\newtheorem{corollary}{Corollary}[section]
\providecommand*{\corollaryautorefname}{Corollary}

% define proposition
\newtheorem{proposition}{Proposition}[section]
\providecommand*{\propositionautorefname}{Proposition}

% define conjecture
\newtheorem{conjecture}{Conjecture}[section]
\providecommand*{\conjectureautorefname}{Conjecture}

\theoremstyle{remark}

% define observation
\newtheorem{observation}{Observation}[section]
\providecommand*{\observationautorefname}{Observation}

% define remark
\newtheorem{remark}{Remark}[section]
\providecommand*{\remarkautorefname}{Remark}

\theoremstyle{definition}

% define example
\newtheorem{example}{Example}[section]
\providecommand*{\exampleautorefname}{Example}

% define definition
\newtheorem{definition}{Definition}[section]
\providecommand*{\definitionautorefname}{Definition}

\newcommand{\blackbox}{\hfill$\blacksquare$}
\usepackage{tikz}
\newcommand*\circled[1]{\tikz[baseline=(char.base)]{
            \node[shape=circle,draw,inner sep=2pt] (char) {#1};}}

\let\lil\lstinputlisting
\usepackage{afterpage}
\usepackage{xcolor}
\hypersetup{colorlinks, allcolors=., urlcolor=blue}

\begin{document}
\maketitle

\newpage
\pagestyle{plain}
\tableofcontents
\newpage
\lstlistoflistings
\newpage
\pagestyle{fancy}

\hypertarget{about-the-author}{%
\section{About the Author}\label{about-the-author}}

\href{http://www.wfindlay.com/}{William ``HousedHorse'' Findlay} has
been playing Age of Empires II since his father introduced it to him
when William was three years old.

He has played Starcraft II at a semi-professional level, reaching rank
number 5 on the North American server. He is also 2100 rated on Age of
Empires II HD.

William is an admin of the Age of Empires Memes Community group on
Facebook, an Age fan page with over 30,000 members.

He works as a teaching assistant in the Computer Science department at
Carleton University in Ottawa, Ontario, Canada. William enjoys scripting
and coding and sees random map scripting as a great way to blend his
passion for programming and his love of Age of Empires II.

\hypertarget{what-is-rms}{%
\section{What is RMS?}\label{what-is-rms}}

Welcome to my \textbf{random map scripting} guide! If you are here, you
probably have some general idea of what an RMS is. If not, don't worry!
I will be explaining all about random map scripting in this section and
all the great things you can do with it.

When you think about custom games in Age of Empires II, typically a
scenario is the first thing that comes to mind. Scenarios are great. You
can do lots of cool things in them. The only issue with a scenario is
that it is the same thing every time. And if a mapmaker does not take
great care, a scenario can easily become too unbalanced for competitive
play.

Age of Empires takes care of this problem with the concept of random
maps. While a lot of games use ``scenarios'' for their competitive
matches, Age of Empires competitive matches are played
\textbf{exclusively} on random maps. By learning how to create random
map scripts of your own, you can create custom map types for players to
enjoy competitively.

Some examples of random maps include Arabia, Arena, Black Forest, Nomad,
and many, many more. These are all examples of maps that come with the
base game. However, if we journey to the \texttt{Custom} section in the
lobby screen, we can see a list of \textbf{custom random maps}. In this
guide, we are going to learn to create maps like these.

\hypertarget{why-is-it-called-a-random-map-script}{%
\subsection{Why is it Called a Random Map
Script?}\label{why-is-it-called-a-random-map-script}}

Age of Empires II uses a random map generator to create its random maps.
A random map generator is an \textbf{algorithm} which takes a random
value called a seed and carries out a series of steps to output a full
randomized map.

By default, the generator will spit out a flat grassy map with no
resources and a town center, scout, and villagers for every player. But
this is super boring. How do we make it do interesting things?

Well, it turns out we can write a text file called a \textbf{random map
script}. This text file is structured in a special way so the random map
generator can read it and use it as a guideline for map generation.

For example, we could set the terrain to be snowy instead of the default
grass and add randomly placed lakes and resources. The generator will
read these instructions and adjust its random output so the map fits our
guidelines. Now we're starting to have something that looks more like a
real map!

\hypertarget{how-to-use-this-guide}{%
\section{How to Use This Guide}\label{how-to-use-this-guide}}

In this section we will detail exactly how to read this guide. We will
be going over some simple conventions we use to make reading the guide
simpler and following that up with some basic
RMS\footnote{Random Map Script.} syntax\footnote{The rules we have to
follow so the generator can read our script.} to get you started and
thinking about what a proper script looks like.

\hypertarget{guide-conventions}{%
\subsection{Guide Conventions}\label{guide-conventions}}

In this guide, we will be showing script examples throughout. The
examples will be shown in the form of code blocks which look like
\autoref{example-block}.

\lil[caption={[Example code block]An example RMS code block.},label={example-block}]{code/3/intro.rms}

Each block will be labeled with a \textbf{listing number}. This is the
number that will be referenced in text. If you see a listing number in
the text, you can take a look nearby for the accompanying sample code.

Code blocks like in \autoref{example-block} will be surrounded by a
black box. To the left of the blocks, you will see line numbers for your
reference. If you are copying the sample code, \textbf{do not copy the
line numbers}. They are just for your reference.

\hypertarget{basic-rms-syntax}{%
\subsection{Basic RMS Syntax}\label{basic-rms-syntax}}

In this subsection, we will learn about some basic syntax to get you
started thinking about what an RMS should look like. Syntax is the way
our script has to be formatted in order to allow the game to read it
properly. Syntax is \textbf{super important} or your script will
\textbf{break horribly} and you will be sad.

\hypertarget{comments}{%
\subsubsection{Comments}\label{comments}}

Comments are bits of text that you can read and understand but the
random map generator ignores completely. It is important to comment your
script so that you can come back and understand it later.

I also like to use comments in my script template so that when I am
making a new script I have easy access to all the constants and
functions I need. I will be going over my template much later in the
guide.

\textbf{So, how do we write a comment?} It's pretty easy. Comments start
with \texttt{/*} and end with \texttt{*/} . The \texttt{/*} and
\texttt{*/} do not need to be on the same line and in this way you can
write a comment that spans multiple lines. A comment can be as short or
as long as you like. See \autoref{comment-example} for an example.

\lil[caption={[Comments]The two types of comments in a random map script. Note the /* and */
which denote the comment's beginning and end. Also note the space between the beginning
and end of a comment and the comment itself.},label={comment-example},float]{code/3/comment-example.rms}

As you can see in \autoref{comment-example}, it's pretty easy to make a
comment. \textbf{But there is one super important thing to watch out
for!} When you write a comment it is \textbf{required} to leave a space
between the \texttt{/*} and the first word and the \texttt{*/} and the
last word. \textbf{If you do not use those spaces your script will not
work}.

\hypertarget{constants}{%
\subsubsection{Constants}\label{constants}}

A constant is a way of binding a \emph{value} to a \emph{name}
\textbf{permanently} in your script. What kind of value do I mean? Every
object and terrain in the game is represented by a \emph{number} that
the random map generator understands. To make it so you don't have to
memorize a list of weird numbers, the game \emph{predefines} many
constants for you.

For example instead of typing \texttt{1234}, I can type \texttt{GRASS}
and the game knows that I'm talking about the grassy terrain.

However, some objects and terrains do not come with a predefined name.
Instead, we have to look up the number and give it a name of our own.
There are a few rules for doing this:

\begin{enumerate}
\def\labelenumi{\arabic{enumi}.}
\tightlist
\item
  The name you choose must not already be assigned to \emph{another
  number} either by you \emph{or} by the game.
\item
  The number you choose must be assigned to an actual object in the
  game. Choosing a random number may cause the game to crash!
\item
  You define your constant as follows: \texttt{\#const\ NAME\ NUMBER}
\end{enumerate}

Suppose for example, we wanted to include a dead tree in our map.
Spooky. We simply do it as shown in \autoref{constex}.

\lil[caption={[Constants]Defining a constant for the dead tree object.},label={constex}]{code/3/deadtree.rms}

There will be a table of \textbf{all} constants with names \emph{and}
without names for your convenience later in this guide. Check the table
of contents for a page number.

\hypertarget{sections}{%
\subsubsection{Sections}\label{sections}}

A random map script is divided into \emph{seven sections}. Specific
parts of your script will go into each section. A section \emph{begins}
with the section name enclosed in \texttt{\textless{}\ \textgreater{}}
and \emph{ends} when you start a new one. See \autoref{sections} for an
example.

Note that the sections are shown in a particular order. You \emph{can}
place the sections in any order you like. However, \textbf{the order in
which I have presented the sections is the order in which the game will
read them}. In order to avoid confusion, I strongly recommend you use
the same order for your sections. This will help you figure why your
script is behaving the way it is.

\lil[caption={[RMS sections]The seven sections of a random map script.},label={sections}]{code/3/sections.rms}

\hypertarget{blocks}{%
\subsubsection{Blocks}\label{blocks}}

\hypertarget{fields}{%
\paragraph{Fields}\label{fields}}

\hypertarget{outline-of-a-script}{%
\section{Outline of a Script}\label{outline-of-a-script}}

\hypertarget{the-preamble}{%
\subsection{The Preamble}\label{the-preamble}}

\hypertarget{player-setup}{%
\subsection{Player Setup}\label{player-setup}}

\hypertarget{land-generation}{%
\subsection{Land Generation}\label{land-generation}}

\hypertarget{elevation-generation}{%
\subsection{Elevation Generation}\label{elevation-generation}}

\hypertarget{terrain-generation}{%
\subsection{Terrain Generation}\label{terrain-generation}}

\hypertarget{connection-generation}{%
\subsection{Connection Generation}\label{connection-generation}}

\hypertarget{object-generation}{%
\subsection{Object Generation}\label{object-generation}}

\hypertarget{player-towns}{%
\subsubsection{Player Towns}\label{player-towns}}

\hypertarget{regicide-support}{%
\paragraph{Regicide Support}\label{regicide-support}}

\hypertarget{arena-style-maps}{%
\subsubsection{Arena Style Maps}\label{arena-style-maps}}

\hypertarget{resources}{%
\subsubsection{Resources}\label{resources}}

\hypertarget{eye-candy}{%
\subsubsection{Eye Candy}\label{eye-candy}}

\hypertarget{constants-1}{%
\section{Constants}\label{constants-1}}

Information for this section based on
\href{http://aok.heavengames.com/blacksmith/showfile.php?fileid=12178}{Zetnus'
RMS Guide}.

The tables are divided into:

\begin{enumerate}
\def\labelenumi{\arabic{enumi}.}
\tightlist
\item
  Buildings
\item
  Units

  \begin{enumerate}
  \def\labelenumii{\roman{enumii})}
  \tightlist
  \item
    Player
  \item
    Gaia
  \end{enumerate}
\item
  Resource Objects
\item
  Terrains/Lands

  \begin{enumerate}
  \def\labelenumii{\roman{enumii})}
  \tightlist
  \item
    Normal Terrain
  \item
    Forest
  \item
    Water
  \end{enumerate}
\end{enumerate}

The columns are divided into:

\begin{enumerate}
\def\labelenumi{\arabic{enumi}.}
\tightlist
\item
  Description

  \begin{itemize}
  \tightlist
  \item
    A brief description of the object/terrain
  \item
    Extra information here
  \end{itemize}
\end{enumerate}

\hypertarget{buildings}{%
\subsection{Buildings}\label{buildings}}

\begin{longtable}[]{@{}lll@{}}
\toprule
\begin{minipage}[b]{0.41\columnwidth}\raggedright
Description\strut
\end{minipage} & \begin{minipage}[b]{0.28\columnwidth}\raggedright
Name\strut
\end{minipage} & \begin{minipage}[b]{0.11\columnwidth}\raggedright
Number\strut
\end{minipage}\tabularnewline
\midrule
\endhead
\begin{minipage}[t]{0.41\columnwidth}\raggedright
Town Center\\
Use with \texttt{set\_place\_for\_every\_player}\strut
\end{minipage} & \begin{minipage}[t]{0.28\columnwidth}\raggedright
TOWN\_CENTER\strut
\end{minipage} & \begin{minipage}[t]{0.11\columnwidth}\raggedright
109\strut
\end{minipage}\tabularnewline
\begin{minipage}[t]{0.41\columnwidth}\raggedright
\strut
\end{minipage} & \begin{minipage}[t]{0.28\columnwidth}\raggedright
\strut
\end{minipage} & \begin{minipage}[t]{0.11\columnwidth}\raggedright
\strut
\end{minipage}\tabularnewline
\begin{minipage}[t]{0.41\columnwidth}\raggedright
Wall\\
Use with \texttt{set\_place\_for\_every\_player} Only works if player
lands are connected\strut
\end{minipage} & \begin{minipage}[t]{0.28\columnwidth}\raggedright
WALL\\
STONE\_WALL\strut
\end{minipage} & \begin{minipage}[t]{0.11\columnwidth}\raggedright
117\strut
\end{minipage}\tabularnewline
\begin{minipage}[t]{0.41\columnwidth}\raggedright
\strut
\end{minipage} & \begin{minipage}[t]{0.28\columnwidth}\raggedright
\strut
\end{minipage} & \begin{minipage}[t]{0.11\columnwidth}\raggedright
\strut
\end{minipage}\tabularnewline
\begin{minipage}[t]{0.41\columnwidth}\raggedright
'\strut
\end{minipage} & \begin{minipage}[t]{0.28\columnwidth}\raggedright
'\strut
\end{minipage} & \begin{minipage}[t]{0.11\columnwidth}\raggedright
'\strut
\end{minipage}\tabularnewline
\begin{minipage}[t]{0.41\columnwidth}\raggedright
'\strut
\end{minipage} & \begin{minipage}[t]{0.28\columnwidth}\raggedright
'\strut
\end{minipage} & \begin{minipage}[t]{0.11\columnwidth}\raggedright
'\strut
\end{minipage}\tabularnewline
\begin{minipage}[t]{0.41\columnwidth}\raggedright
'\strut
\end{minipage} & \begin{minipage}[t]{0.28\columnwidth}\raggedright
'\strut
\end{minipage} & \begin{minipage}[t]{0.11\columnwidth}\raggedright
'\strut
\end{minipage}\tabularnewline
\begin{minipage}[t]{0.41\columnwidth}\raggedright
'\strut
\end{minipage} & \begin{minipage}[t]{0.28\columnwidth}\raggedright
'\strut
\end{minipage} & \begin{minipage}[t]{0.11\columnwidth}\raggedright
'\strut
\end{minipage}\tabularnewline
\begin{minipage}[t]{0.41\columnwidth}\raggedright
'\strut
\end{minipage} & \begin{minipage}[t]{0.28\columnwidth}\raggedright
'\strut
\end{minipage} & \begin{minipage}[t]{0.11\columnwidth}\raggedright
'\strut
\end{minipage}\tabularnewline
\begin{minipage}[t]{0.41\columnwidth}\raggedright
'\strut
\end{minipage} & \begin{minipage}[t]{0.28\columnwidth}\raggedright
'\strut
\end{minipage} & \begin{minipage}[t]{0.11\columnwidth}\raggedright
'\strut
\end{minipage}\tabularnewline
\begin{minipage}[t]{0.41\columnwidth}\raggedright
'\strut
\end{minipage} & \begin{minipage}[t]{0.28\columnwidth}\raggedright
'\strut
\end{minipage} & \begin{minipage}[t]{0.11\columnwidth}\raggedright
'\strut
\end{minipage}\tabularnewline
\begin{minipage}[t]{0.41\columnwidth}\raggedright
'\strut
\end{minipage} & \begin{minipage}[t]{0.28\columnwidth}\raggedright
'\strut
\end{minipage} & \begin{minipage}[t]{0.11\columnwidth}\raggedright
'\strut
\end{minipage}\tabularnewline
\begin{minipage}[t]{0.41\columnwidth}\raggedright
'\strut
\end{minipage} & \begin{minipage}[t]{0.28\columnwidth}\raggedright
'\strut
\end{minipage} & \begin{minipage}[t]{0.11\columnwidth}\raggedright
'\strut
\end{minipage}\tabularnewline
\bottomrule
\end{longtable}

\hypertarget{units}{%
\subsection{Units}\label{units}}

\hypertarget{player-units}{%
\subsubsection{Player Units}\label{player-units}}

\begin{longtable}[]{@{}lll@{}}
\toprule
\begin{minipage}[b]{0.41\columnwidth}\raggedright
Description\strut
\end{minipage} & \begin{minipage}[b]{0.28\columnwidth}\raggedright
Name\strut
\end{minipage} & \begin{minipage}[b]{0.11\columnwidth}\raggedright
Number\strut
\end{minipage}\tabularnewline
\midrule
\endhead
\begin{minipage}[t]{0.41\columnwidth}\raggedright
'\strut
\end{minipage} & \begin{minipage}[t]{0.28\columnwidth}\raggedright
'\strut
\end{minipage} & \begin{minipage}[t]{0.11\columnwidth}\raggedright
'\strut
\end{minipage}\tabularnewline
\begin{minipage}[t]{0.41\columnwidth}\raggedright
'\strut
\end{minipage} & \begin{minipage}[t]{0.28\columnwidth}\raggedright
'\strut
\end{minipage} & \begin{minipage}[t]{0.11\columnwidth}\raggedright
'\strut
\end{minipage}\tabularnewline
\begin{minipage}[t]{0.41\columnwidth}\raggedright
'\strut
\end{minipage} & \begin{minipage}[t]{0.28\columnwidth}\raggedright
'\strut
\end{minipage} & \begin{minipage}[t]{0.11\columnwidth}\raggedright
'\strut
\end{minipage}\tabularnewline
\begin{minipage}[t]{0.41\columnwidth}\raggedright
'\strut
\end{minipage} & \begin{minipage}[t]{0.28\columnwidth}\raggedright
'\strut
\end{minipage} & \begin{minipage}[t]{0.11\columnwidth}\raggedright
'\strut
\end{minipage}\tabularnewline
\begin{minipage}[t]{0.41\columnwidth}\raggedright
'\strut
\end{minipage} & \begin{minipage}[t]{0.28\columnwidth}\raggedright
'\strut
\end{minipage} & \begin{minipage}[t]{0.11\columnwidth}\raggedright
'\strut
\end{minipage}\tabularnewline
\begin{minipage}[t]{0.41\columnwidth}\raggedright
'\strut
\end{minipage} & \begin{minipage}[t]{0.28\columnwidth}\raggedright
'\strut
\end{minipage} & \begin{minipage}[t]{0.11\columnwidth}\raggedright
'\strut
\end{minipage}\tabularnewline
\begin{minipage}[t]{0.41\columnwidth}\raggedright
'\strut
\end{minipage} & \begin{minipage}[t]{0.28\columnwidth}\raggedright
'\strut
\end{minipage} & \begin{minipage}[t]{0.11\columnwidth}\raggedright
'\strut
\end{minipage}\tabularnewline
\begin{minipage}[t]{0.41\columnwidth}\raggedright
'\strut
\end{minipage} & \begin{minipage}[t]{0.28\columnwidth}\raggedright
'\strut
\end{minipage} & \begin{minipage}[t]{0.11\columnwidth}\raggedright
'\strut
\end{minipage}\tabularnewline
\begin{minipage}[t]{0.41\columnwidth}\raggedright
'\strut
\end{minipage} & \begin{minipage}[t]{0.28\columnwidth}\raggedright
'\strut
\end{minipage} & \begin{minipage}[t]{0.11\columnwidth}\raggedright
'\strut
\end{minipage}\tabularnewline
\begin{minipage}[t]{0.41\columnwidth}\raggedright
'\strut
\end{minipage} & \begin{minipage}[t]{0.28\columnwidth}\raggedright
'\strut
\end{minipage} & \begin{minipage}[t]{0.11\columnwidth}\raggedright
'\strut
\end{minipage}\tabularnewline
\begin{minipage}[t]{0.41\columnwidth}\raggedright
'\strut
\end{minipage} & \begin{minipage}[t]{0.28\columnwidth}\raggedright
'\strut
\end{minipage} & \begin{minipage}[t]{0.11\columnwidth}\raggedright
'\strut
\end{minipage}\tabularnewline
\bottomrule
\end{longtable}

\hypertarget{gaia-units}{%
\subsubsection{Gaia Units}\label{gaia-units}}

\begin{longtable}[]{@{}lll@{}}
\toprule
\begin{minipage}[b]{0.41\columnwidth}\raggedright
Description\strut
\end{minipage} & \begin{minipage}[b]{0.28\columnwidth}\raggedright
Name\strut
\end{minipage} & \begin{minipage}[b]{0.11\columnwidth}\raggedright
Number\strut
\end{minipage}\tabularnewline
\midrule
\endhead
\begin{minipage}[t]{0.41\columnwidth}\raggedright
'\strut
\end{minipage} & \begin{minipage}[t]{0.28\columnwidth}\raggedright
'\strut
\end{minipage} & \begin{minipage}[t]{0.11\columnwidth}\raggedright
'\strut
\end{minipage}\tabularnewline
\begin{minipage}[t]{0.41\columnwidth}\raggedright
'\strut
\end{minipage} & \begin{minipage}[t]{0.28\columnwidth}\raggedright
'\strut
\end{minipage} & \begin{minipage}[t]{0.11\columnwidth}\raggedright
'\strut
\end{minipage}\tabularnewline
\begin{minipage}[t]{0.41\columnwidth}\raggedright
'\strut
\end{minipage} & \begin{minipage}[t]{0.28\columnwidth}\raggedright
'\strut
\end{minipage} & \begin{minipage}[t]{0.11\columnwidth}\raggedright
'\strut
\end{minipage}\tabularnewline
\begin{minipage}[t]{0.41\columnwidth}\raggedright
'\strut
\end{minipage} & \begin{minipage}[t]{0.28\columnwidth}\raggedright
'\strut
\end{minipage} & \begin{minipage}[t]{0.11\columnwidth}\raggedright
'\strut
\end{minipage}\tabularnewline
\begin{minipage}[t]{0.41\columnwidth}\raggedright
'\strut
\end{minipage} & \begin{minipage}[t]{0.28\columnwidth}\raggedright
'\strut
\end{minipage} & \begin{minipage}[t]{0.11\columnwidth}\raggedright
'\strut
\end{minipage}\tabularnewline
\begin{minipage}[t]{0.41\columnwidth}\raggedright
'\strut
\end{minipage} & \begin{minipage}[t]{0.28\columnwidth}\raggedright
'\strut
\end{minipage} & \begin{minipage}[t]{0.11\columnwidth}\raggedright
'\strut
\end{minipage}\tabularnewline
\begin{minipage}[t]{0.41\columnwidth}\raggedright
'\strut
\end{minipage} & \begin{minipage}[t]{0.28\columnwidth}\raggedright
'\strut
\end{minipage} & \begin{minipage}[t]{0.11\columnwidth}\raggedright
'\strut
\end{minipage}\tabularnewline
\begin{minipage}[t]{0.41\columnwidth}\raggedright
'\strut
\end{minipage} & \begin{minipage}[t]{0.28\columnwidth}\raggedright
'\strut
\end{minipage} & \begin{minipage}[t]{0.11\columnwidth}\raggedright
'\strut
\end{minipage}\tabularnewline
\begin{minipage}[t]{0.41\columnwidth}\raggedright
'\strut
\end{minipage} & \begin{minipage}[t]{0.28\columnwidth}\raggedright
'\strut
\end{minipage} & \begin{minipage}[t]{0.11\columnwidth}\raggedright
'\strut
\end{minipage}\tabularnewline
\begin{minipage}[t]{0.41\columnwidth}\raggedright
'\strut
\end{minipage} & \begin{minipage}[t]{0.28\columnwidth}\raggedright
'\strut
\end{minipage} & \begin{minipage}[t]{0.11\columnwidth}\raggedright
'\strut
\end{minipage}\tabularnewline
\begin{minipage}[t]{0.41\columnwidth}\raggedright
'\strut
\end{minipage} & \begin{minipage}[t]{0.28\columnwidth}\raggedright
'\strut
\end{minipage} & \begin{minipage}[t]{0.11\columnwidth}\raggedright
'\strut
\end{minipage}\tabularnewline
\bottomrule
\end{longtable}

\hypertarget{resource-objects}{%
\subsection{Resource Objects}\label{resource-objects}}

\begin{longtable}[]{@{}lll@{}}
\toprule
\begin{minipage}[b]{0.41\columnwidth}\raggedright
Description\strut
\end{minipage} & \begin{minipage}[b]{0.28\columnwidth}\raggedright
Name\strut
\end{minipage} & \begin{minipage}[b]{0.11\columnwidth}\raggedright
Number\strut
\end{minipage}\tabularnewline
\midrule
\endhead
\begin{minipage}[t]{0.41\columnwidth}\raggedright
'\strut
\end{minipage} & \begin{minipage}[t]{0.28\columnwidth}\raggedright
'\strut
\end{minipage} & \begin{minipage}[t]{0.11\columnwidth}\raggedright
'\strut
\end{minipage}\tabularnewline
\begin{minipage}[t]{0.41\columnwidth}\raggedright
'\strut
\end{minipage} & \begin{minipage}[t]{0.28\columnwidth}\raggedright
'\strut
\end{minipage} & \begin{minipage}[t]{0.11\columnwidth}\raggedright
'\strut
\end{minipage}\tabularnewline
\begin{minipage}[t]{0.41\columnwidth}\raggedright
'\strut
\end{minipage} & \begin{minipage}[t]{0.28\columnwidth}\raggedright
'\strut
\end{minipage} & \begin{minipage}[t]{0.11\columnwidth}\raggedright
'\strut
\end{minipage}\tabularnewline
\begin{minipage}[t]{0.41\columnwidth}\raggedright
'\strut
\end{minipage} & \begin{minipage}[t]{0.28\columnwidth}\raggedright
'\strut
\end{minipage} & \begin{minipage}[t]{0.11\columnwidth}\raggedright
'\strut
\end{minipage}\tabularnewline
\begin{minipage}[t]{0.41\columnwidth}\raggedright
'\strut
\end{minipage} & \begin{minipage}[t]{0.28\columnwidth}\raggedright
'\strut
\end{minipage} & \begin{minipage}[t]{0.11\columnwidth}\raggedright
'\strut
\end{minipage}\tabularnewline
\begin{minipage}[t]{0.41\columnwidth}\raggedright
'\strut
\end{minipage} & \begin{minipage}[t]{0.28\columnwidth}\raggedright
'\strut
\end{minipage} & \begin{minipage}[t]{0.11\columnwidth}\raggedright
'\strut
\end{minipage}\tabularnewline
\begin{minipage}[t]{0.41\columnwidth}\raggedright
'\strut
\end{minipage} & \begin{minipage}[t]{0.28\columnwidth}\raggedright
'\strut
\end{minipage} & \begin{minipage}[t]{0.11\columnwidth}\raggedright
'\strut
\end{minipage}\tabularnewline
\begin{minipage}[t]{0.41\columnwidth}\raggedright
'\strut
\end{minipage} & \begin{minipage}[t]{0.28\columnwidth}\raggedright
'\strut
\end{minipage} & \begin{minipage}[t]{0.11\columnwidth}\raggedright
'\strut
\end{minipage}\tabularnewline
\begin{minipage}[t]{0.41\columnwidth}\raggedright
'\strut
\end{minipage} & \begin{minipage}[t]{0.28\columnwidth}\raggedright
'\strut
\end{minipage} & \begin{minipage}[t]{0.11\columnwidth}\raggedright
'\strut
\end{minipage}\tabularnewline
\begin{minipage}[t]{0.41\columnwidth}\raggedright
'\strut
\end{minipage} & \begin{minipage}[t]{0.28\columnwidth}\raggedright
'\strut
\end{minipage} & \begin{minipage}[t]{0.11\columnwidth}\raggedright
'\strut
\end{minipage}\tabularnewline
\begin{minipage}[t]{0.41\columnwidth}\raggedright
'\strut
\end{minipage} & \begin{minipage}[t]{0.28\columnwidth}\raggedright
'\strut
\end{minipage} & \begin{minipage}[t]{0.11\columnwidth}\raggedright
'\strut
\end{minipage}\tabularnewline
\bottomrule
\end{longtable}

\hypertarget{terrainslands}{%
\subsection{Terrains/Lands}\label{terrainslands}}

\hypertarget{normal-terrain}{%
\subsubsection{Normal Terrain}\label{normal-terrain}}

\begin{longtable}[]{@{}lll@{}}
\toprule
\begin{minipage}[b]{0.41\columnwidth}\raggedright
Description\strut
\end{minipage} & \begin{minipage}[b]{0.28\columnwidth}\raggedright
Name\strut
\end{minipage} & \begin{minipage}[b]{0.11\columnwidth}\raggedright
Number\strut
\end{minipage}\tabularnewline
\midrule
\endhead
\begin{minipage}[t]{0.41\columnwidth}\raggedright
'\strut
\end{minipage} & \begin{minipage}[t]{0.28\columnwidth}\raggedright
'\strut
\end{minipage} & \begin{minipage}[t]{0.11\columnwidth}\raggedright
'\strut
\end{minipage}\tabularnewline
\begin{minipage}[t]{0.41\columnwidth}\raggedright
'\strut
\end{minipage} & \begin{minipage}[t]{0.28\columnwidth}\raggedright
'\strut
\end{minipage} & \begin{minipage}[t]{0.11\columnwidth}\raggedright
'\strut
\end{minipage}\tabularnewline
\begin{minipage}[t]{0.41\columnwidth}\raggedright
'\strut
\end{minipage} & \begin{minipage}[t]{0.28\columnwidth}\raggedright
'\strut
\end{minipage} & \begin{minipage}[t]{0.11\columnwidth}\raggedright
'\strut
\end{minipage}\tabularnewline
\begin{minipage}[t]{0.41\columnwidth}\raggedright
'\strut
\end{minipage} & \begin{minipage}[t]{0.28\columnwidth}\raggedright
'\strut
\end{minipage} & \begin{minipage}[t]{0.11\columnwidth}\raggedright
'\strut
\end{minipage}\tabularnewline
\begin{minipage}[t]{0.41\columnwidth}\raggedright
'\strut
\end{minipage} & \begin{minipage}[t]{0.28\columnwidth}\raggedright
'\strut
\end{minipage} & \begin{minipage}[t]{0.11\columnwidth}\raggedright
'\strut
\end{minipage}\tabularnewline
\begin{minipage}[t]{0.41\columnwidth}\raggedright
'\strut
\end{minipage} & \begin{minipage}[t]{0.28\columnwidth}\raggedright
'\strut
\end{minipage} & \begin{minipage}[t]{0.11\columnwidth}\raggedright
'\strut
\end{minipage}\tabularnewline
\begin{minipage}[t]{0.41\columnwidth}\raggedright
'\strut
\end{minipage} & \begin{minipage}[t]{0.28\columnwidth}\raggedright
'\strut
\end{minipage} & \begin{minipage}[t]{0.11\columnwidth}\raggedright
'\strut
\end{minipage}\tabularnewline
\begin{minipage}[t]{0.41\columnwidth}\raggedright
'\strut
\end{minipage} & \begin{minipage}[t]{0.28\columnwidth}\raggedright
'\strut
\end{minipage} & \begin{minipage}[t]{0.11\columnwidth}\raggedright
'\strut
\end{minipage}\tabularnewline
\begin{minipage}[t]{0.41\columnwidth}\raggedright
'\strut
\end{minipage} & \begin{minipage}[t]{0.28\columnwidth}\raggedright
'\strut
\end{minipage} & \begin{minipage}[t]{0.11\columnwidth}\raggedright
'\strut
\end{minipage}\tabularnewline
\begin{minipage}[t]{0.41\columnwidth}\raggedright
'\strut
\end{minipage} & \begin{minipage}[t]{0.28\columnwidth}\raggedright
'\strut
\end{minipage} & \begin{minipage}[t]{0.11\columnwidth}\raggedright
'\strut
\end{minipage}\tabularnewline
\begin{minipage}[t]{0.41\columnwidth}\raggedright
'\strut
\end{minipage} & \begin{minipage}[t]{0.28\columnwidth}\raggedright
'\strut
\end{minipage} & \begin{minipage}[t]{0.11\columnwidth}\raggedright
'\strut
\end{minipage}\tabularnewline
\bottomrule
\end{longtable}

\hypertarget{forest}{%
\subsubsection{Forest}\label{forest}}

\begin{longtable}[]{@{}lll@{}}
\toprule
\begin{minipage}[b]{0.41\columnwidth}\raggedright
Description\strut
\end{minipage} & \begin{minipage}[b]{0.28\columnwidth}\raggedright
Name\strut
\end{minipage} & \begin{minipage}[b]{0.11\columnwidth}\raggedright
Number\strut
\end{minipage}\tabularnewline
\midrule
\endhead
\begin{minipage}[t]{0.41\columnwidth}\raggedright
'\strut
\end{minipage} & \begin{minipage}[t]{0.28\columnwidth}\raggedright
'\strut
\end{minipage} & \begin{minipage}[t]{0.11\columnwidth}\raggedright
'\strut
\end{minipage}\tabularnewline
\begin{minipage}[t]{0.41\columnwidth}\raggedright
'\strut
\end{minipage} & \begin{minipage}[t]{0.28\columnwidth}\raggedright
'\strut
\end{minipage} & \begin{minipage}[t]{0.11\columnwidth}\raggedright
'\strut
\end{minipage}\tabularnewline
\begin{minipage}[t]{0.41\columnwidth}\raggedright
'\strut
\end{minipage} & \begin{minipage}[t]{0.28\columnwidth}\raggedright
'\strut
\end{minipage} & \begin{minipage}[t]{0.11\columnwidth}\raggedright
'\strut
\end{minipage}\tabularnewline
\begin{minipage}[t]{0.41\columnwidth}\raggedright
'\strut
\end{minipage} & \begin{minipage}[t]{0.28\columnwidth}\raggedright
'\strut
\end{minipage} & \begin{minipage}[t]{0.11\columnwidth}\raggedright
'\strut
\end{minipage}\tabularnewline
\begin{minipage}[t]{0.41\columnwidth}\raggedright
'\strut
\end{minipage} & \begin{minipage}[t]{0.28\columnwidth}\raggedright
'\strut
\end{minipage} & \begin{minipage}[t]{0.11\columnwidth}\raggedright
'\strut
\end{minipage}\tabularnewline
\begin{minipage}[t]{0.41\columnwidth}\raggedright
'\strut
\end{minipage} & \begin{minipage}[t]{0.28\columnwidth}\raggedright
'\strut
\end{minipage} & \begin{minipage}[t]{0.11\columnwidth}\raggedright
'\strut
\end{minipage}\tabularnewline
\begin{minipage}[t]{0.41\columnwidth}\raggedright
'\strut
\end{minipage} & \begin{minipage}[t]{0.28\columnwidth}\raggedright
'\strut
\end{minipage} & \begin{minipage}[t]{0.11\columnwidth}\raggedright
'\strut
\end{minipage}\tabularnewline
\begin{minipage}[t]{0.41\columnwidth}\raggedright
'\strut
\end{minipage} & \begin{minipage}[t]{0.28\columnwidth}\raggedright
'\strut
\end{minipage} & \begin{minipage}[t]{0.11\columnwidth}\raggedright
'\strut
\end{minipage}\tabularnewline
\begin{minipage}[t]{0.41\columnwidth}\raggedright
'\strut
\end{minipage} & \begin{minipage}[t]{0.28\columnwidth}\raggedright
'\strut
\end{minipage} & \begin{minipage}[t]{0.11\columnwidth}\raggedright
'\strut
\end{minipage}\tabularnewline
\begin{minipage}[t]{0.41\columnwidth}\raggedright
'\strut
\end{minipage} & \begin{minipage}[t]{0.28\columnwidth}\raggedright
'\strut
\end{minipage} & \begin{minipage}[t]{0.11\columnwidth}\raggedright
'\strut
\end{minipage}\tabularnewline
\begin{minipage}[t]{0.41\columnwidth}\raggedright
'\strut
\end{minipage} & \begin{minipage}[t]{0.28\columnwidth}\raggedright
'\strut
\end{minipage} & \begin{minipage}[t]{0.11\columnwidth}\raggedright
'\strut
\end{minipage}\tabularnewline
\bottomrule
\end{longtable}

\hypertarget{water}{%
\subsubsection{Water}\label{water}}

\begin{longtable}[]{@{}lll@{}}
\toprule
\begin{minipage}[b]{0.41\columnwidth}\raggedright
Description\strut
\end{minipage} & \begin{minipage}[b]{0.28\columnwidth}\raggedright
Name\strut
\end{minipage} & \begin{minipage}[b]{0.11\columnwidth}\raggedright
Number\strut
\end{minipage}\tabularnewline
\midrule
\endhead
\begin{minipage}[t]{0.41\columnwidth}\raggedright
'\strut
\end{minipage} & \begin{minipage}[t]{0.28\columnwidth}\raggedright
'\strut
\end{minipage} & \begin{minipage}[t]{0.11\columnwidth}\raggedright
'\strut
\end{minipage}\tabularnewline
\begin{minipage}[t]{0.41\columnwidth}\raggedright
'\strut
\end{minipage} & \begin{minipage}[t]{0.28\columnwidth}\raggedright
'\strut
\end{minipage} & \begin{minipage}[t]{0.11\columnwidth}\raggedright
'\strut
\end{minipage}\tabularnewline
\begin{minipage}[t]{0.41\columnwidth}\raggedright
'\strut
\end{minipage} & \begin{minipage}[t]{0.28\columnwidth}\raggedright
'\strut
\end{minipage} & \begin{minipage}[t]{0.11\columnwidth}\raggedright
'\strut
\end{minipage}\tabularnewline
\begin{minipage}[t]{0.41\columnwidth}\raggedright
'\strut
\end{minipage} & \begin{minipage}[t]{0.28\columnwidth}\raggedright
'\strut
\end{minipage} & \begin{minipage}[t]{0.11\columnwidth}\raggedright
'\strut
\end{minipage}\tabularnewline
\begin{minipage}[t]{0.41\columnwidth}\raggedright
'\strut
\end{minipage} & \begin{minipage}[t]{0.28\columnwidth}\raggedright
'\strut
\end{minipage} & \begin{minipage}[t]{0.11\columnwidth}\raggedright
'\strut
\end{minipage}\tabularnewline
\begin{minipage}[t]{0.41\columnwidth}\raggedright
'\strut
\end{minipage} & \begin{minipage}[t]{0.28\columnwidth}\raggedright
'\strut
\end{minipage} & \begin{minipage}[t]{0.11\columnwidth}\raggedright
'\strut
\end{minipage}\tabularnewline
\begin{minipage}[t]{0.41\columnwidth}\raggedright
'\strut
\end{minipage} & \begin{minipage}[t]{0.28\columnwidth}\raggedright
'\strut
\end{minipage} & \begin{minipage}[t]{0.11\columnwidth}\raggedright
'\strut
\end{minipage}\tabularnewline
\begin{minipage}[t]{0.41\columnwidth}\raggedright
'\strut
\end{minipage} & \begin{minipage}[t]{0.28\columnwidth}\raggedright
'\strut
\end{minipage} & \begin{minipage}[t]{0.11\columnwidth}\raggedright
'\strut
\end{minipage}\tabularnewline
\begin{minipage}[t]{0.41\columnwidth}\raggedright
'\strut
\end{minipage} & \begin{minipage}[t]{0.28\columnwidth}\raggedright
'\strut
\end{minipage} & \begin{minipage}[t]{0.11\columnwidth}\raggedright
'\strut
\end{minipage}\tabularnewline
\begin{minipage}[t]{0.41\columnwidth}\raggedright
'\strut
\end{minipage} & \begin{minipage}[t]{0.28\columnwidth}\raggedright
'\strut
\end{minipage} & \begin{minipage}[t]{0.11\columnwidth}\raggedright
'\strut
\end{minipage}\tabularnewline
\begin{minipage}[t]{0.41\columnwidth}\raggedright
'\strut
\end{minipage} & \begin{minipage}[t]{0.28\columnwidth}\raggedright
'\strut
\end{minipage} & \begin{minipage}[t]{0.11\columnwidth}\raggedright
'\strut
\end{minipage}\tabularnewline
\bottomrule
\end{longtable}

\hypertarget{example-scripts}{%
\section{Example Scripts}\label{example-scripts}}

\hypertarget{my-script-template}{%
\subsection{My Script Template}\label{my-script-template}}

\href{https://raw.githubusercontent.com/HousedHorse/hh-rms-aoe2/master/template.rms}{This
is the template I use to create all my scripts}. I find it very easy to
start with this file and edit it to my liking. Clicking the link at the
beginning of this paragraph will allow you to copy and paste it.

I have included some extra constants for convenience as well as comments
listing popular alternatives to boars, wolves, sheep, forests, etc.

The default settings are for a standard Arabia map. You can change them
however you like.

\hypertarget{my-github}{%
\subsection{My GitHub}\label{my-github}}

Check out my \href{https://github.com/HousedHorse/hh-rms-aoe2}{GitHub
repository} which contains all the RMS scripts I have ever written.

Caution: some of these scripts are old and crappy!

\hypertarget{my-steam-workshop}{%
\subsection{My Steam Workshop}\label{my-steam-workshop}}

You can also check out my
\href{https://steamcommunity.com/id/housedhorse/myworkshopfiles/?appid=221380}{Steam
Workshop page}.


\end{document}
